%%%
% set up document type
%%%
\documentclass[12pt]{article}

%%%
% declare all packages
%%%
\usepackage[left=25mm, top=20mm, right=25mm, bottom=30mm,nohead,nofoot]{geometry} 

\usepackage[T2A]{fontenc}
\usepackage[utf8]{inputenc}
\usepackage[english, russian]{babel}

\usepackage{graphics, graphicx}

\usepackage{url}
\usepackage{hyperref}

\usepackage{amssymb,latexsym} 
\usepackage{MnSymbol}
\usepackage{mathrsfs}

\usepackage[nottoc,numbib]{tocbibind}
\usepackage{float}
\usepackage{listings}
\usepackage{multirow}
\usepackage{hhline}
\usepackage{delarray}

\usepackage{color,colortbl}

% \usepackage{verbatim}
%%%
% document settings
%%%
\setcounter{tocdepth}{4}
\graphicspath{ {./pic/} }

\renewcommand{\listoffigures}{\begingroup  % add number to list of graphics
\tocsection
\tocfile{\listfigurename}{lof}
\endgroup}
\renewcommand{\listoftables}{\begingroup  % add number to list of tables
\tocsection
\tocfile{\listtablename}{lot}
\endgroup}

\newcommand{\RomanNum}[1]
    {\MakeUppercase{\romannumeral #1}}

%******************************************************************
%******************************************************************
\begin{document}

\begin{titlepage}
	\center
		Санкт-Петербургский Политехнический 
		университет \\ Петра Великого\\
		Институт прикладной математики и механики
		\\ \textbf{Высшая школа прикладной математики и вычислительной физики}

	\vfill ~
	\textbf{
		\\ \large РАСЧЁТНОЕ ЗАДАНИЕ №2
	}
	\\	на тему 
	\\ "Поиск оптимального управления"
	\\ по дисциплине
	\\ "Математическая теория управления"
    \\ \textbf{вариант №3}
	\vfill ~
    
    
    \begin{flushleft}
    \underline{Выполнил:}  \hspace{\fill} студент гр.3630102/60101 \textbf{Лансков.Н.В.} \linebreak[2]
	\underline{Руководитель:} \hspace{\fill} доцент \textbf{Суханов А.А.} \\ 
    \end{flushleft}
    

\vfill

{\large}	Санкт-Петербург
\\ 2020
\end{titlepage}

%%%
% Table of conetnts 
%%%

\tableofcontents
\pagebreak
% \listoffigures
% \listoftables
% \newpage

%%%
% Text
%%%
\section{Постановка задачи}

Дана система:
\begin{equation}\label{eq1}
 \begin{cases}
  \ddot x - \omega_0^2x = u\\
  x(0) = x_0 \\
  \dot x(0) = v_0
 \end{cases}
\end{equation}

Требуется найти такое оптимальное управление с обратной связью, при котором будет достигать минимума интегрально квадратичный функционал качества О:

\begin{equation}
 J = \int\limits_{0}^{\infty}(x^2 + q\dot x^2 + ru^2)dt \rightarrow min
\end{equation}

Кроме того, требуется определить значение минимума этого функционала при значениях параметров:

$$
 \begin{cases}
  q = \dfrac{1}{3} ,\ r = \dfrac{1}{9} \\
  \omega_0 = 1.5 \\
  x_0 = 1 \\
  v_0 = 8
 \end{cases}
$$



\section{Решение}

Сначала переведём систему (\ref{eq1}) в пространство состояний. Для этого воспользуемся общей методикой перевода системы в пространство состояний:

\begin{enumerate}
 \item $ x = x_1 \Rightarrow C = (1, 0)$
 \item $ D(\dot x_1) = u + \omega_0^2x_1$
\end{enumerate}
Положив $\dot x_1 = x_2$, запишем систему

$$
 \begin{cases}
  \dot x_2 = u + \omega_0^2x_1 \\
  x_2(0) = \dot x_1(0) = \dot x(0) = v_0 
 \end{cases} 
$$

Таким образом, система (\ref{eq1}) перепишется в пространстве состояний в следующем виде:

\begin{equation}\label{eq3}
 \begin{cases}
  \dot x_1 = x_2 \\
  \dot x_2 = u + \omega_0^2x_1 \\
  x_1(0) = x_0 \\
  x_2(0) = v_0
 \end{cases}
\end{equation}

\pagebreak
Найдём вид матриц $A, B$:

\begin{equation}
 A = 
 \begin{pmatrix}
  0 & 1\\
  \omega_0^2 & 0 \\
 \end{pmatrix}
 , \ B = 
 \begin{pmatrix}
  0 \\
  1
 \end{pmatrix}
\end{equation}

В общем виде интегрально-квадратичный функционал качества записывается так:

\begin{equation}
 J = \int\limits_{0}^{\infty} (x^TQx + u^TRu)dt
\end{equation}

Определим вид матриц $Q, R$:

\begin{equation}
 Q = 
 \begin{pmatrix}
  1 & 0\\
  0 & q \\
 \end{pmatrix}
 , \ R = r
\end{equation}

Известно, что оптимальное управление $u^{opt} = -K^*x$, причём коэффициенты $K^* = (K_1, K_2)$ могут быть вычислены из следлующей системы уравнений:

\begin{equation}
 K^* = R^{-1}B^TP^*
\end{equation}

где $P^*$ - неортрицательное решение матричного уравнения Риккатти-Лурье. В таком случае, минимум функционала $J_{min} = x_0^TPx_0$.
Для применения теоремы о существовании и единственности неотрицательного решения уравнения Риккатти-Лурье необходимо проверить систему (\ref{eq3}) на полную управляемость, также необходимо удостовериться в том, что $Q \geq 0, R > 0$. Для проверки системы на полную управляемость воспользуемся теоремой Калмана и проверим выполнение условия:

\begin{equation}
 rankA_u = rank(B, AB) = rank
 \begin{pmatrix}
  1 & 0 \\
  0 & 1 \\
 \end{pmatrix}
 = 2
\end{equation}

Значит, условия теоремы выполнены.

\subsection{Программный расчёт с использованием python}

Воспользуемся встроенной функцией (K, P) = control.lqr(A, B, Q, R). Эта функция в качестве возвращаемых значений выдаёт вектор коэффициентов $K^*$ и матрицу $P^*$.

В результате получаем:
\begin{equation}
 K^* = (6, 3.87298335), \
 P^* = 
 \begin{pmatrix}
  1.61374306 & 0.66666667 \\
  0.66666667 & 0.43033148 \\
 \end{pmatrix}
\end{equation}

\begin{equation}
J_{min} = 39.82162463 
\end{equation}

\begin{equation}
 u^{opt} = -K^*x = -6x_1 - 3.87298335x_2 = -6x - 3.87298335\dot x
\end{equation}

\subsection{Ручной расчёт}

Найдём решение уравнения Риккатти-Лурье:

\begin{equation}
 A^TP + PA - PBR^{-1}B^TP + Q = 0s
\end{equation}

\begin{eqnarray}
  \begin{pmatrix}
   0 & \omega_0^2 \\
   1 & 0
  \end{pmatrix}
  \cdot
  \begin{pmatrix}
    p_{11} & p_{12} \\
    p_{11} & p_{22} \\
  \end{pmatrix}
  + 
  \begin{pmatrix}
    p_{11} & p_{12} \\
    p_{11} & p_{22} \\
  \end{pmatrix}
  \cdot
  \begin{pmatrix}
   0 & 1 \\
   \omega_0^2 & 0
  \end{pmatrix}
  - \nonumber \\
  \begin{pmatrix}
    p_{11} & p_{12} \\
    p_{11} & p_{22} \\
  \end{pmatrix}
  \cdot
  \begin{pmatrix}
    0 \\ 1
  \end{pmatrix}
  \cdot \dfrac{1}{r}
  \cdot 
  \begin{pmatrix}
    0 & 1 \\
  \end{pmatrix}
  \cdot
  \begin{pmatrix}
    p_{11} & p_{12} \\
    p_{11} & p_{22} \\
  \end{pmatrix}
  +  
  \begin{pmatrix}
    1 & 0 \\
    0 & q \\
  \end{pmatrix}
  = 0
\end{eqnarray}

Иначе,
\begin{equation}
 \begin{pmatrix}
  2\omega_0^2p_{12} - \dfrac{p_{12}^2}{r} + 1 &
  \omega_0^2p_{22} + p_{11} - \dfrac{p_{12}p_{22}}{r} \\
  \omega_0^2p_{22} + p_{11} - \dfrac{p_{12}p_{22}}{r} &
  2p_{12} - \dfrac{p_{22}}{r} + q
 \end{pmatrix}
  = 0
\end{equation}

Найдём решение системы:

\begin{equation}
 \begin{cases}
  2\omega_0^2p_{12} - \dfrac{p_{12}^2}{r} + 1 = 0\\
  \omega_0^2p_{22} + p_{11} - \dfrac{p_{12}p_{22}}{r} = 0 \\
  2p_{12} - \dfrac{p_{22}}{r} + q = 0
 \end{cases}
\end{equation}

В результате получаем следующие соотношения:

\begin{equation}
 \begin{cases}
  p_{12} = \dfrac{2r\omega_0^2 \pm \sqrt{4r^2\omega_0^4 + 4r}}{2}\\
  p_{22} = \pm \sqrt{2rp_{12} + qr} \\
  p_{11} = \dfrac{p_{12}p_{22}}{r} - \omega_0^2p_{22}
 \end{cases}
\end{equation}

Подставив заданные в условии значения констант, получаем:

\begin{equation}
 P_1 = 
 \begin{pmatrix}
  1.61374306 & 0.66666667 \\
  0.66666667 & 0.43033148 \\
 \end{pmatrix}, \
 P_2 = 
 \begin{pmatrix}
  -1.61374306 & 0.66666667 \\
  0.66666667 & -0.43033148 \\
 \end{pmatrix}
\end{equation}

Следовательно, $P^* = P_1$. Тогда:

\begin{equation}
 K^* = (R^{-1}B^TP^*) = (6, 3.87298335)
\end{equation}

\begin{equation}
J_{min} = 39.82162463 
\end{equation}

\begin{equation}
 u^{opt} = -K^*x = -6x_1 - 3.87298335x_2 = -6x - 3.87298335\dot x
\end{equation}

\section{Результаты}
В результате работы было найдено оптимальное управление с обратной связью для поставленной задачи(программно и `вручную`), а также определено значение минимума функционала качества.

\section{Приложения}
Расчётное задание выполнено с использованием системы вёрстки Latex.
\end{document}

